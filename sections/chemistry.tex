% ==================================================================
%                               ХИМИЯ
% ==================================================================
\section*{Химия}

\TaskBlock{Задача 1}{30 баллов}{Тема: сплавы в современном мире.}

\ConditionBlock

В современном мире технологии и инженерии сплавы металлов играют важнейшую роль. 
Эти материалы сочетают свойства различных металлов, создавая уникальные характеристики, которые невозможно достичь с помощью чистых металлов. 
Сплавы металлов – это не просто комбинации элементов, а инновационные решения для множества проблем, стоящих перед инженерами и учеными. 
Использование сплавов дает возможность создавать материалы с уникальным набором физико-химических характеристик, 
чтобы оптимизировать производственные процессы и повысить эффективность изделий. 
Это причина того, почему они так широко распространены в самых разнообразных областях: от строительства до высоких технологий.

Силумин (сплав на основе алюминия и кремния) представляет собой важный материал, сочетающий в себе прочность, устойчивость и легкость, 
что обеспечивает его широкое применение в различных областях промышленности и дизайна. Благодаря прочности, низкому весу, 
коррозионной стойкости силумин используется в автомобилестроении, авиастроении, кораблестроении, в космической промышленности.

% --- (Рис 1) ---
\begin{figure}[H]
    \centering
    \includegraphics[width=0.7\textwidth]{images/image1.jpg}
\end{figure}

Для анализа образца силумина, содержащего алюминий и кремний, массой 30 г его
растворили в 400 г 15\% ‐ного раствора едкого натра, при этом выделился газ, объемом 38,4
л (н.у.). Определите массовую долю алюминия в сплаве (в процентах). Число округлите до
целых.

\SolutionBlock
{ % Группа для локальных настроек
\setlength{\parindent}{0pt} % Полностью убираем отступы слева в этом блоке
\setlength{\parskip}{0.5em} % Небольшой отступ между абзацами

Составим уравнения реакций: \\[0.2em]
$\displaystyle \ce{2Al + 2NaOH + 6H2O -> 2Na[Al(OH)4] + 3H2 \uparrow}$ \\
$\displaystyle \ce{Si + 2NaOH + H2O -> Na2SiO3 + 2H2 \uparrow}$ \\
$\displaystyle n(\ce{H2}) = \frac{38{,}4 \text{ л}}{22{,}4 \text{ л/моль}} = 1{,}714 \text{ моль}$ \\
Если $n(\ce{Al}) = x \text{ моль}, n(\ce{Si}) = y \text{ моль}$, тогда: \\[0.2em]
$\displaystyle 
\renewcommand{\arraystretch}{1.2}
\begin{cases}
    27x + 28y = 30 \\
    1{,}5x + 2y = 1{,}714
\end{cases}
$

$\displaystyle x = 1 \text{ моль}, \quad y = 0{,}107 \text{ моль}$
$\displaystyle m(\ce{Al}) = n \cdot M = 1 \cdot 27 = 27 \text{ г}$ \\
$\displaystyle \omega(\ce{Al}) = 27/30 = 0{,}9 \text{ или } 90 \%$

} % Конец группы

\AnswerBlock
90.