% ==================================================================
%                               ГЕОГРАФИЯ
% ==================================================================
\section*{География}

\TaskBlock{Задача 1. Дельты рек}{30 баллов}{Тема: региональные физико-географические особенности; география гидроэнергетики.}

\ConditionBlock

На космических снимках представлены дельты крупных рек России. Соотнесите фото и названия рек.

\vspace{1em}

\renewcommand{\arraystretch}{1.5}
\setlength{\tabcolsep}{10pt}

\begin{longtable}{|c|m{\dimexpr\linewidth-4.5cm}|c|}
    \hline
    \rowcolor{TableGray} 
    \textbf{№} & \centering \textbf{Снимок} & \textbf{Река} \tabularnewline
    \hline
    \endfirsthead 

    % --- Строки таблицы ---
    1 & 
    \centering \vspace{1mm}
    \includegraphics[width=6.5cm]{images/image2.png} \\
    \textbf{Рис. 2} \vspace{1mm}
    & Лена \tabularnewline
    \hline
    
    2 & 
    \centering \vspace{1mm}
    \includegraphics[width=6.5cm]{images/image3.png} \\
    \textbf{Рис. 3} \vspace{1mm}
    & Волга \tabularnewline
    \hline
    
    3 & 
    \centering \vspace{1mm}
    \includegraphics[width=6.5cm]{images/image4.png} \\
    \textbf{Рис. 4} \vspace{1mm}
    & Хатанга \tabularnewline
    \hline
    
    4 & 
    \centering \vspace{1mm}
    \includegraphics[width=6.5cm]{images/image5.png} \\
    \textbf{Рис. 5} \vspace{1mm}
    & Обь \tabularnewline
    \hline

\end{longtable}

\vspace{1em}

\AnswerBlock
Снимок 1 — Лена; снимок 2 — Обь; снимок 3 — Хатанга; снимок 4 — Волга.